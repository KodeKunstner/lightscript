\documentclass[12pt]{article} 
\usepackage{a4} 
\usepackage{makeidx}
\usepackage[utf8]{inputenc}
\usepackage{textcomp}
\usepackage{amsmath}
\usepackage{amssymb}
\usepackage{amsthm}
\usepackage{graphicx} 
\usepackage{verbatim} 
\usepackage{fancyhdr}
\usepackage{listings} 
\usepackage{url}
\frenchspacing
\makeindex
\pagestyle{plain}
\newcommand{\Oh}[0]{ \mathcal{O} }
%\addtolength{\voffset}{-30pt}
%\addtolength{\textheight}{00pt}

\setcounter{tocdepth}{3}

\title{Design and implementation of a mobile scripting language}

\author{
  Rasmus Jensen\footnote{
    sumsar@solsort.dk
  }
} 

% Remember 
%    \index terms
%    update template as well as document when structural updates.
\date{July 2008}

\begin{document}

\maketitle

%\begin{abstract}
%\end{abstract}


%\tableofcontents


%%%%%%%%%%%%%%%%%%%%%%%%
% intro, purpose

The topic of the project is the design and implementation of a scripting language.
This is both to create a practical language, 
and also a focus for exploration of programming language theory.

The following sections will first elaborate on the topic of the project

\section{The language}






%%%%%%%%%%%%%%%%%%%%%%%%
% learning goals

%%%%%%%%%%%%%%%%%%%%%%%%
% structure of the thesis


%%%%%%%%%%%%%%%%%%%%%%%%
% related literature

%%%%%%%%%%%%%%%%%%%%%%%%
% my background

\nocite{sicp}
\appendix

\addcontentsline{toc}{section}{Litteratur}
\bibliography{bibliography}
\bibliographystyle{alpha}

%\addcontentsline{toc}{section}{Indeks}
%\printindex

\end{document}

