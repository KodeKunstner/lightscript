\documentclass[11pt]{article} 
\usepackage{a4}
\usepackage{makeidx}
%\usepackage[danish]{babel} 
\usepackage[utf8]{inputenc}
\usepackage{textcomp}
\usepackage{amsmath}
\usepackage{amssymb}
\usepackage{amsthm}
\usepackage{graphicx}
\usepackage{verbatim}
\usepackage{fancyhdr}
\usepackage{listings}
\usepackage[colorlinks,pagebackref]{hyperref}
\usepackage{backref}
\usepackage{url}
\frenchspacing
\makeindex
\pagestyle{plain}
\newcommand{\Oh}[0]{ \mathcal{O} }
%\addtolength{\voffset}{-50pt}
%\addtolength{\textheight}{60pt}

\setcounter{tocdepth}{3}

\title{Design and implementation of \\ a small scripting language}

\author{Rasmus Jensen\footnote{\url{rasmusjensen@solsort.dk}}
} 

\begin{document}
\maketitle


%%%%%%%%%%%%%%%%%%%%%%%%
% intro, purpose

The topic of the project is to design and implement a scripting language.
This is both to create a practical tool, 
and also a focus for exploration of programming language theory.

The learning goals are to learn about programming on limited devices and to learn about programming language design and implementation. 
Through the project, I should be able to evaluate and choose programming language features and implementation techniques, design a full language and implement it on top of different platforms with different limitations.

The language should be a portable embedable high level scripting language, with a low memory footprint. 
Portable, such that it even runs in webbrowsers, without special plugins, and on low-end mobile devices. 
Embedable, such that it is very easy to include within, and interface with, applications in other languages.
High level, such that it supports features such as higher order functions and metaprogramming.
Low memory footprint, such that the compiler and runtime system should be in the magnitude of tens, rather than hundreds, of kilobytes.
The approach to the design should be of minimalistic and pragmatic.

\section{Literature}

For the basics there is the book ``Programming Language Design Concepts''\cite{programming-language-design-concepts}. 
The books [TODO:real title for the tiger book]\cite{tigerbook} 
and ``Basics of Compiler Design''\cite{basics-of-compiler-design} gives the basics for compiler implementation, and 
``Structure and Intepretation of Computer Programs''\cite{sicp} also has a bit on evaluation and compilation.
%The book ``Concepts, Techniques, and Models of Computer Programming''\cite{bookvanroy} is also usefull 

There is a report on the Rabbit\cite{rabbit} Scheme compiler which gives an interesting insight into continuations and the use of a small core language.
  \cite{essence-cps} has a followup on continuation passing style, and points in directions of single static assignment and A normal form.
There is an article\cite{stack-vs-register} that covers performance characteristics of stack versus register machines. 
This is also touched by \cite{case-for-vm} which also has some details of the performance of different implementations of dispatch.
Hoare\cite{hoare-hints} has some hints on programming language design, 
and Graham\cite{graham-five-questions} also has some nice ideas about language design. 
Ousterhout\cite{ousterhout-scripting} gives some motivation for scripting languages.

For details of potential target languages, the EcmaScript specification\cite{ecma262}, the documentation of the 
Java virtual machine (JVM)\cite{jvm}  
and mobile edition\cite{j2me}, 
the C programming language\cite{knr}, 
the specification of python\cite{python} are of high importance. 

An interesting taget for integration is the DrScheme environment\cite{drscheme}, which is a customisable introductory development environment.

Languages and virtual machines, that could be especially interesting to look more into are Tcl\cite{tcl-tk}, 
Lua\cite{lua5}, 
Hecl\footnote{Tcl inspired language on Java Mobile Edition}\cite{Hecl}, 
Scheme\cite{r5rs}, 
Parrot\footnote{Virtual machine for scripting languages}\cite{parrot}, 
kvm\footnote{A JVM for mobile devices}\cite{kvm}

%(Things to explore futher is the Erlang concurrency model, A-normal form, tracing jit compiler.)

\section{Project plan}

The project consists of the following tasks:
\begin{itemize}
\item First revision of the language: survey, design, write specification and simple intepreter. (2 months)
\item Implement the language on top of EcmaScript. Write some sample applications in the language and API. Document the implementation and possibly revise the language design. (1 month)
\item Implement the language on top of Java Mobile Edition. This might include a simple virtual machine. Write more sample applications and API. Document the implementation and possibly revise the language design and earlier implementations. (1 month)
\item Implement the language on top of C. This includes a virtual machine. Document the implementation and possibly revise the language design and earlier implementations. (1-2 months)
\item Write report on current state of project, more sample code, documentation, experiments, benchmark, based on the progress so far. 
\item If time permits, integrate with the DrScheme environment. (1-2 months)
\item Finalise report (1 month).
\end{itemize}

If the project progresses slower than scheduled, the DrScheme integration can be skipped. If the project progresses faster than scheduled, some of the following can be added:
\begin{itemize}
\item Experiments with different compiler and virtual machine optimisations (1 month)
\item Implementation of the language on top of Python (1 month)
\item A survey of partial evaluation, and an implementation for the language if feasible (2-? months)
\end{itemize}



\begin{figure}
\begin{tabular}{|p{6cm}|p{12cm}|}
\hline
\hline
\end{tabular}
\end{figure}

%Suggested platforms would be: EcmaScript and Java Mobile Edition and C. 
%EcmaScript is the dominant language within webbrowsers. It has dynamic type system, and some versions only have floating point arithmetics.
%Java Mobile Edition is the dominating platform for applications for mobile phones. It is a shrunk down version of Java, and some versions have only integer arithmetics and is limited to 64KB application.
%C is the main language for system programming and embedded development. It does not have garbage collection and can be viewed as a high level portable assembler.

%The language design might have a lisp-inspired syntax, arrays and hashtables as main data structures, a simple macro system, python inspired module system, direct access to compilation and loading of code, and good interoperability with the JSON data exchange format.

%The content of this synopsis is the following: The first section elaborates on why we need a new language and where the focus is. The second section contains the learning goals for the project. The third section describes how the final report should be structured. The fourth section is about my background for writing this project.  A bibliography with some literature for the project is appended at the end.


%The language I want is a highly portable scripting language, with a low memory footprint. 
%A language that 
%A language that supports higher order functions and metaprogramming.
%A language that is easy to embed and connect with native code.
%
%The design will probably end up with syntax similar to lisp, semantics similar to ecmascript and limitations similar to those of jave mobile edition.


%My background is that I am returning to computer science, after being away for a while.
%My recent computer science studies have mainly been within computer science education, data compression and information science.
%I have had some algorithmic and programming language theory, but it is long time ago.

%\appendix

\addcontentsline{toc}{section}{References}
\bibliography{bibliography}
\bibliographystyle{plain}

%\addcontentsline{toc}{section}{Indeks}
%\printindex

\end{document}
