This document is a drafty sketch of the preliminary ideas for the language. 



A main focus of the language design is to minimise memory usage, including footprint of the compiler and run-time system.
At the same time it should be expressive and easy to learn.

Naming: It should be a small language, and a name for it could be yoctolanguage, or yolan, where yocto- is one of the smallest SI-units.

\section{Language design}

\subsection{Syntax}
\subsubsection{Isomorph syntax tree and source code}
It should be easy to map between different representations of the code.
In the initial versions this would be 

The motivation for this is that it allows easier refactoring of the code.
In a long term perspective this will also be practical HERE!

\subsubsection{The actual syntax}


\subsection{Compiler and macro system}
\subsection{Native function interface}

\subsection{Module system}

\section{Features not in first version}
\subsection{Object system}
\subsection{Threading}
\subsection{Partial evaluation}
\subsection{Features}
\subsubsection{Platform independence}
\subsubsection{Macro system / reconfigurable language}
\subsubsection{Close integration with JSON}
\subsubsection{Tiny memory footprint}
\subsubsection{Easy integration with native language/platform}

\subsection{Flaws}
\subsubsection{Dynamic typing}
\subsubsection{Slowness}
\subsubsection{Lots of Irritating Silly Parenthesis}

