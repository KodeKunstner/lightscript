\chapter{Yolan}
\label{yolan}
\index{Yolan}
\section{Design choices}
\begin{comment}
null-is-false
dynamic scope

... every expression yields a value, possibly false. Yes, you can bind a variable to the result of a false condition.

... lazy ffi.

\subsection{* Syntax}
A lisp-like syntax is choosen in order to simplify the parser and thereby reduce the code footprint.
Square brackets are used instead of parenthesis in order to indicate that the lists are not lisp-like cons-lists, but arrays. 

\section{* Implementation details}
\subsection{* AST rewriting}

\end{comment}
\section{Language specification}

\subsection{Syntax}
The syntax is similar to lisp, in the sense that function applications are written as lists, where the first element in the list is the function to be applied.
Programs consist of lists, where each element is either a list, a variable name, or a literal.

\subsubsection{Function application / lists}
Function applications are written in lisp-style as lists. The first element in the list is the function to be applied.
Lists are enclosed within square brackest \verb|[|$\cdots$\verb|]|, and may be nested. The elements within the lists are seperated by whitespaces. 
As lists are the notation for function applications, every list must have at least one element, which is the function to be applied.

\subsubsection{Variable names}
A variable name is a sequence of characters. The possible characters are letters, numbers, the symbols \verb"!#$'()*+-,-./:<=>?@\^_`{|}~", and any unicode symbol with an unicode alue of 127 or higher. The first character in the name of a variable must be non-numeric.

\subsubsection{Integer literals}

Integers are written as a sequence of digits (\verb|0123456789|). Only base 10 input is possible and only non-negative numbers can be written as literals. Negative integers must be generated by subtraction.

\subsubsection{Comments and whitespaces}
Characters with a unicode value of 32 or less are regarded as whitespaces. This includes the usual space, tab, newline, and line-feed. Whitespaces are used to seperate list elements, and are discarded during parsing. A comment must be preceded by a whitespace, starts with a semicolon \verb|;| and continues until the end of the line. Comments are discarded during parsing.

\subsection{Builtin functions}
The builtin functions are listed in the following sections. As the language is designed for embedding in other applications, there are no standard input/output function, file access, network, etc. as these might not be present or differ significantly between target devices/platforms.

In the following, the function names are written with {\tt fixed width} font, and the parameters are written in $cursive$. Parameters can be any expression, and are named according to their type or function: $num$s are expressions that should evaluate to numbers, $exp$s are expressions that may be optionally evaluated (e.g. in {\tt if}), $val$s are expressions will be evaluated, $string$s are expressions that should evaluate to a string, and so on.

\subsubsection{Variables}
\subsubsection*{\tt{[set }$name$ $value$\tt{]}}
Evaluate $value$ and let $name$ refer to the result.

\subsubsection*{\tt{[locals [}$name_1 \cdots name_n$\tt{]} $expr_1 \cdots expr_n$\tt{]}}
Let $name_1 \cdots name_n$ be local variables in $expr_1 \cdots expr_n$: First save the values corresponding to $name_1 \cdots name_n$, then evaluate $expr_1 \cdots expr_n$, next restore the values of $name_1 \cdots name_n$ and finally return the result of the evaluation of $expr_n$.

\subsubsection{Conditionals and logic}
\subsubsection*{\tt{[if }$cond$ $expr_1$ $expr_2$\tt{]}}
Evaluate $cond$ and if the result is non-$nil$ then evaluate and return $expr_1$, else evaluate and return $expr_2$.

\subsubsection*{\tt{[not }$cond$\tt{]}}
If $cond$ is $nil$ return $true$ else return $nil$.

\subsubsection*{\tt{[and }$expr_1$ $expr_2$\tt{]}}
Evaluate $expr_1$ and if it is non-$nil$, evaluate and return the value of $expr_2$, else return $nil$.

\subsubsection*{\tt{[or }$expr_1$ $expr_2$\tt{]}}
Evaluate $expr_1$ and if it is non-$nil$ return its value, else evaluate and return the value of $expr_2$.

\subsubsection{Repetition and sequencing}
\subsubsection*{\tt{[repeat }$num$ $expr_1 \cdots expr_n$\tt{]}}
Evaluate $expr_1 \cdots expr_n$ $num$ number of times ($num$ must evaluate to a number). The result is the last execution of $expr_n$, or $nil$ if no expressions were evaluated, i.e. $num \leq 0$.

\subsubsection*{\tt{[foreach }$name$ $iterator$ $expr_1 \cdots expr_n$\tt{]}}
For every value from the $iterator$, bind it to the local $name$ and evaluate $expr_1 \cdots expr_n$. The result of the evaluation is the last executed $expr_n$ or $nil$ if no expressions were evaluated. $name$ is a local variable, and is thus saved before the loop, and restored afterwards.

\subsubsection*{\tt{[while }$cond$ $expr_1 \cdots expr_n$\tt{]}}
While $cond$ evaluates to non-$nil$, evaluate $expr_1 \cdots expr_n$, and return the value of the last $expr_n$ or $nil$ if no expressions were evaluated.


\subsubsection*{\tt{[do }$expr_1 \cdots expr_n$\tt{]}}
Evaluate $expr_1 \cdots expr_n$ and return the result of $expr_n$.


\subsubsection{Functions}
\subsubsection*{\tt{[lambda [}$name_1 \cdots name_n$\tt{]} $expr_1 \cdots expr_n$\tt{]}}
Create a new anonymous function, with the parameters $name_1\cdots name_n$. Application of the function will bind its arguments to local variables $name_1\cdots name_n$, evaluate $expr_1\cdots expr_n$ and return $expr_n$.
\subsubsection*{\tt{[defun [}$name_{function}$ $name_1 \cdots name_n$\tt{]} $expr_1 \cdots expr_n$\tt{]}}
Create a new function, and bind it to the variable $name_{function}$. It is equivalent to {\tt{[set }}$name_{function}${\tt{ [lambda [}}$name_1 \cdots name_n${\tt{]}} $expr_1 \cdots expr_n${\tt{]]}}.

\subsubsection*{\tt{[apply }$function$ $param_1 \cdots param_n$\tt{]}}
Apply the $function$ to the parameters $param_1\cdots param_n$. The difference between this and the usual function application {\tt{[}$function$ $param_1\cdots param_n$\tt{]}} is that that \verb|apply| allows $function$ to change between invocations, whereas the usual function application assumes that $function$ is static to be able to optimise it during runtime.

\subsubsection{Integer operations}
\subsubsection*{\tt{[+ }$num_1$ $num_2$\tt{]}}
Calculate the sum of two integers.
\subsubsection*{\tt{[- }$num_1$ $num_2$\tt{]}}
Calculate the difference of two integers, the result is $num_2$ subtracted from $num_1$.
\subsubsection*{\tt{[* }$num_1$ $num_2$\tt{]}}
Calculate the product of two integers.
\subsubsection*{\tt{[/ }$num_1$ $num_2$\tt{]}}
Integer division, $num_1$ is divided by $num_2$.
\subsubsection*{\tt{[\% }$num_1$ $num_2$\tt{]}}
Returns the remainder of dividing $num_1$ by $num_2$.

\subsubsection{Type predicates}
\subsubsection*{\tt{[is-integer }$val$\tt{]}}
Returns $true$ if $val$ is an integer.
\subsubsection*{\tt{[is-string }$val$\tt{]}}
Returns $true$ if $val$ is a string.
\subsubsection*{\tt{[is-list }$val$\tt{]}}
Returns $true$ if $val$ is a list.
\subsubsection*{\tt{[is-dictionary }$val$\tt{]}}
Returns $true$ if $val$ is a dictionary.
\subsubsection*{\tt{[is-iterator }$val$\tt{]}}
Returns $true$ if $val$ is a iterator.

\subsubsection{Polymorphic functions}
\subsubsection*{\tt{[equals }$val_1$ $val_2$\tt{]}}
Compare $val_1$ to $val_2$ and return $true$ if they are the same, or $nil$ if they are different. $val_1$ and $val_2$ must have the same type, and should either be integers or strings.
\subsubsection*{\tt{[is-empty }$val$\tt{]}}
Returns $true$ if a list, dictionary or iterator does not have any elements. Else it returns $nil$.
\subsubsection*{\tt{[put }$container$ $position$ $value$\tt{]}}
Store a value into a a list or a dictionary. If it is a list, the $position$ must be an integer in the range $0,1, \cdots, ${\tt{[size }}$container${\tt{]}}$-1$.
If it is a dictionary, the position must be a string or an integer. An entry is deleted from a dictionary by storing $nil$ as the $value$.
\subsubsection*{\tt{[get }$container$ $position$\tt{]}}
Retrieve a value from a list or a dictionary. It has the same constraints on $position$ as with \verb|put|. Retrieving an uninitialised entry from a dictionary yields $nil$.
\subsubsection*{\tt{[random }$val$\tt{]}}
If $val$ is an integer, return a random number in the range $0,1, \cdots, val -1$. If $val$ is a list, pick a random value from the list.
\subsubsection*{\tt{[size }$val$\tt{]}}
Return the length of a string, the number of values in a list, or the number of entries in a dictionary.
\subsubsection*{\tt{[< }$val_1$ $val_2$\tt{]}}
Compares $val_1$ with $val_2$. If $val_1$ and $val_2$ are an integers, return $true$ if $val_1$ is strictly less than $val_2$ and otherwise $nil$.
If $val_1$ and $val_2$ are strings, do a lexicographical comparison and return $true$ if $val_1$ comes strictly before $val_2$, and otherwise $nil$.
\subsubsection*{\tt{[<= }$num_1$ $num_2$\tt{]}}
Compares $val_1$ with $val_2$. If $val_1$ and $val_2$ are an integers, return $true$ if $val_1$ is less than or equal to $val_2$ and otherwise $nil$.
If $val_1$ and $val_2$ are strings, do a lexicographical comparison and return $true$ if they are equal or $val_1$ comes before $val_2$, and otherwise $nil$.

\subsubsection{String functions}
\subsubsection*{\tt{[stringjoin }$val_1\cdots val_n$\tt{]}}
Create a string by concatenating $val_1\cdots val_n$.
If $val_i$ is an integer or a list, it is converted to a string.
A list is converted to a string by concatenating its elements, as if {\tt stringjoin} were called with the list elements as parameters.

\subsubsection*{\tt{[substring }$string$ $num_{begin}$ $num_{end}$\tt{]}}
Create a substring from a string, starting inclusively at character position $num_{begin}$ and ending exclusively at character position $num_{end}$. The positions starts counting at $0$, so thus {\tt{[substring }$string$ $0$ \tt{[size }$string$\tt{]]}} is the entire string.

\subsubsection{List functions}
\subsubsection*{\tt{[list }$val_1\cdots val_n$\tt{]}}
Create a new list, containing the elements $val_1\cdots val_n$.

\subsubsection*{\tt{[resize }$list$ $num$\tt{]}}
Change the size of the $list$ to be $num$ values. 
If the new size is larger than the previous, new values will be added to the end of the list, and they are initialised to be nil. If it is smaller, then the list will be truncated at the end. The list is returned.

\subsubsection*{\tt{[push }$list$ $val$\tt{]}}
Push the value $val$ at the end of the $list$. The size of the list grows by one, and the last element is now $val$.

\subsubsection*{\tt{[pop }$list$\tt{]}}
Remove the element from the end list. The function returns that element, and reduces the size of the list by one.

\subsubsection{Dictionary functions}
\subsubsection*{\tt{[dict }$key_1$  $val_1$ $\cdots$ $key_n$ $val_n$\tt{]}}
Create a new dictionary with $n$ entries, where $key_1$ maps to $val_1$ and so forth.

\subsubsection{Iterator functions}
\subsubsection*{\tt{[keys }$dictionary$\tt{]}}
Create a new iterator across the keys of a dictionary.
\subsubsection*{\tt{[values }$container$\tt{]}}
Create a new iterator across the values of either a dictionary or a list.
\subsubsection*{\tt{[get-next }$iterator$\tt{]}}
Get the next element from the iterator, or nil if the iterator is empty.

\subsubsection{Debugging}
\subsubsection*{\tt{[log }$string$\tt{]}}
Logs the message from $string$, possibly ignored if debugging is disabled.
\subsubsection*{\tt{[assert }$string$ $val$\tt{]}}
Halt the execution with error message $string$ if $val$ is nil, possibly ignored if debugging is disabled.


\section{Developers guide}
Yolan is a minimal scripting language implemented on Java.
It allows scripting to be added to application, with a minimal overhead on the size of the JAR file -- which is the most limiting factor on low end mobile devices. The features and limitations of Yolan are:
\begin{itemize}
\item A scripting language, with support for higher order functions
\item Support for loading of code at run time
\item Operates directly on standard Java classes, such as \verb|java.lang.Integer|, \verb|java.util.Hashtable| and \verb|java.util.Stack|
\item Runs on Java Micro Edition/J2ME, and requires only CLDC 1.0/MIDP 1.0
\item Adds approximately 5KB to the size of the optimised JAR file. Only single-threaded/non-reentrant, to achive the small size
\item Parses and executes one expression at a time, allowing interactive programming, and implying that the entire program need not to be in memory at once
\item Interpreted - code can be entered directly on the device, not needing an extra step of compilation, and thus it is also suitable for scriptable configuration files, user scripts etc.
\end{itemize}

The implementation of Yolan consists of a single class \verb|Yolan| with the actual implementation, and an interface \verb|Function| which is what a class need to implement in order to be callable from Yolan.
While Yolan only has a single classfile for the implementation, in order to reduce the JAR file size, it consists several logical classes: a parser, a single static runtime, and Yolan instantiated objects are Yolan code that can be evaluated, i.e. delayed computations.

Having a single runtime reduces memory usage, but also limits applications to only execute a single script, and only having a single execution context, at a time. The reduction of memory usage comes from that refererences to the execution context can be hard coded, and thus the delayed computations does not have to carry a reference to the context. There are also a memory reduction due to that less code is needed and the class for the context can be joined into the main class file, as static properties.

The \verb|Function| interface consist of a function that takes an array of Yolan objects -- delayed computations -- as parameter, and the return a value. In this sense Java objects callable from Yolan are essentially lazy functions, and themselves responsible for evaluating their arguments.

\subsection{Getting Started}
The core method of a Yolan object is the \verb|value()| method which evaluates the code the Yolan object represents, and returns the result. This method may throw \verb|Exeception|s as well as \verb|Error|s if the code it represents has faults, so if we are executing user code, or want to be robust against errors in scripts, the Yolan evaluation should be surrounded by a \verb|catch(Throwable)|.

Yolan objects are created with the static \verb|readExpression| method that parses the next Yolan expression from an input stream. So if we want to create a simple interactive interpreter, we can write: \begin{lstlisting}
class Main {
    public static void main(String [] args) throws java.io.IOException {
        Yolan yl = Yolan.readExpression(System.in);
        while(yl != null) {
            try {
                System.out.println("Result: " + yl.value().toString());
            } catch(Throwable yolanError) {
                System.out.println("Error: " + yolanError.toString());
            }
            yl = Yolan.readExpression(System.in);
        }
    }
}\end{lstlisting} 

This code could be saved in a file called Main.java, placed in a directory with Yolan.class and Function.class, and then compiled and executed by executing \verb|javac Main.java| and \verb|java Main|.

Notice that the input stream \verb|System.in| can be replaced with any input stream, so the same basic idea for can be used for evaluating files, programs as strings within the application, or even as streams across the network, where Yolan could work as a shell for remote scripting/controlling an application.

If we want to execute an entire stream, there is a short hand builtin method for doing that: \verb|eval|. For example:
\begin{lstlisting}
class Main {
    public static void main(String [] args) throws java.io.IOException {
        Yolan.eval(new FileInputStream(new File("script.yl")));
    }
}
\end{lstlisting}
This code opens the file "script.yl", and evaluate all the expressions within it. 
\verb|eval| throws away the results of the individual expressions and does not print them,
so the above code is only useful if we have added some user defined functions to Yolan that allows it to do something practical.

\subsubsection{Adding functions to the runtime}
This section describes how to make Java code callable from Yolan.
While the builtin Yolan functions supports basic data structures etc., there is no built in way to do input/output from yolan that is platform dependent: Java Standard Edition supports files, where Java Micro Edition has a record store, and user interfaces ranges between Midlets, Applets, graphical applications, and text standard-in/out.
So when the language needs communicate with the user, or work on the state of the host application, 
some functionality needs to be added, which is most easily done by adding functions to the runtime.

The \verb|Function| interface is the way to do that. To implement the interface a single function \verb|apply|, that takes an array of Yolan objects as parameter, and returns an object is required.
Notice that the parameters are passed lazily, e.g. they are only evaluated when the called function chooses to evaluate them, so we need to call the \verb|value()|-function of the Yolan objects when we want the actual value.
Execution of the Yolan object may also have side effects, so whether, and the number of times, the \verb|value()| is called matters.
In order to add a new Java function to be callable from the runtime, the method \verb|Yolan.addFunction| takes a name as a string and a \verb|Function| as parameters, and binds the name to the function. As an example the following code makes a new function, println, available to the runtime. This function takes one argument, which it prints out to the standard output:
\begin{lstlisting}
class PrintingFunction implements Function {
    Object apply(Yolan args[]) {
        System.out.println(args[0].value());
    }
}
class Main {
    public static void main(String [] args) throws java.io.IOException {
        Yolan.addFunction("println", new PrintingFunction);
        Yolan.eval(new FileInputStream(new File("script.yl")));
    }
}
\end{lstlisting} 
The above program reads and evaluates the file script.yl, with an augmented runtime which also has the println function.

\subsubsection{Values and types}
The builtin types in Yolan are mapped to Java classes for easier interoperability,
so lists are implemented as java.util.Stack, dictionaries are implemented as java.util.Hashtable, strings are implemented as java.lang.String, nil/false are implemented as the value null, integers are implemented as java.lang.Integer, and iterators are implemented as java.util.Enumeration. 
Operations on those data types are just as the native builtin types. 

Any Java object can be passed around within Yolan, so adding support for new data types is just a question of adding functions that work on those data types.

\subsubsection{Functions defined withinin Yolan}
When a user defines a function within Yolan, they are instances of the Yolan class. 
Before calling such a function, the number of arguments can be found using the \verb|nargs| method.
If the Yolan object is not a callable user defined function, the result of \verb|nargs()| is -1, which thus can be used to check if a Yolan object is a callable function.
The function is then applied with the \verb|apply| method, which takes the arguments to the function as arguments, for example:
\begin{lstlisting}
...
    // evaluation some yolan object that yields a function
    Yolan function = yl.value();
    // ensure that it is a function and it takes two arguments
    if(function.nargs() == 2) {
        // apply the function 
        result = function.apply(arg1, arg2);
    } else ...
....
\end{lstlisting}
The apply method is defined from zero, up to three arguments. If there is a need for an apply method with more arguments, they are simple to add, see page~\pageref{source-yolan-apply} for the implementation details. There is also a general apply method, that takes an array of arguments as argument.

\subsubsection{Modifying the runtime}

In order for the scripting language to be practical, it should be able to work and share data with the host application. 
Of course this can be done with functions, and evaluation, as described above, but an additional connection with the language can be added by accessing the variables defined, and used, by the running scripts.
For this there are three functions: \verb|Yolan.resolveVar|, \verb|Yolan.getVar|, and \verb|Yolan.setVar|.

When a value is accessed, this is done through a handle, which is found with \verb|resolveVar|. This handle can then be used for reading and writing the variable. The motivation for the handle is that it takes time to lookup what a variable name, so this computation can be done once for each variable that needs to be accessed, and then additional accesses to the resolved variable are significant faster. The \verb|resolveVar| function takes the variable name as a string parameter, and return the handle, which is an integer. If the variable does not exist in the runtime, space is allocated for it.

With a handle, it is then possible to set the value of a variable with \verb|setVar|. For example setting the variable foo to 42 can be done with:
\begin{lstlisting}
    Yolan.setVar(Yolan.resolveVar("foo"), new Integer(42));
\end{lstlisting} 
and similarly the variable can be read with \verb|getVar|:
\begin{lstlisting}
    Object result = Yolan.getVar(Yolan.resolveVar("foo"));
\end{lstlisting}

If it the variable is commonly accessed, it saves time to cache the handle across calls, as follows:
\begin{lstlisting}
class Class {
    int fooHandle;
    Class() {
        fooHandle = Yolan.resolveVar("foo");
    }

    int someMethod() {
        ... perhaps some scripts modifying foo is executed ...
        Object foo = Yolan.getVar(fooHandle);
        ...
    }

    void otherMethod() {
        ... 
        Yolan.setVar(fooHandle, "A literal value or some variable");
        ...
    }
}
\end{lstlisting}
When defining functions, as described earlier, it is actually the same that is happening: the function is encapsulated in a Yolan object and added to the runtime as with \verb|setVar|.

\subsubsection{Resetting the runtime and saving space}
When the scripting language is only used in some part of the application, it can be pratical to be able to unload its runtime data in order to save memory. 
For this there are two utility functions \verb|Yolan.wipe()| and \verb|Yolan.reset()|.

\verb|Yolan.wipe()| sets all references in the runtime are set to zero, allowing data to be garbage collected.
When the runtime has been wiped, Yolan expressions can no longer be evaluated, and trying to evaluate them will yield errors. 

\verb|Yolan.reset()| resets the runtime: all variable handles are invalidated, all variables are removed from the runtime, and only the builtin functions are added. Existing Yolan expressions are invalid, and evaluation of them may lead to unexpected behavior. User defined functions and variables need to be re-added.
Resetting is necessesary before scripts are executed after a \verb|Yolan.wipe()|.
It can also be practical when multiple scripts are run, one after each other, and they must not mess up the runtime for each other.

\subsection{More on adding functions to the runtime}
The naive approach for adding functions to the runtime would be to create a new class implementing the \verb|Function| interface for each function. This adds significantly to the JAR-file
.
If the code size is critical, this can often be reduced by combining the multiple functions into a single class, for example via a switch dispatch as shown below:
\begin{lstlisting}
class ManyFunction implements Function {
    int id;
    ManyFunction(int id) {
        this.id = id;
    }
    Object apply(Yolan args[]) {
        switch(id) {
            case 0: // first function
                    ....
                break;
            case 1: // second function
                    ....
                break;
            case 2: // third function
                    ....
                break;
            ....
            default:
                throw SomeKindOfError();
    }
    static void register() {
        Yolan.addFunction(new ManyFunction(0), "firstFunction");
        Yolan.addFunction(new ManyFunction(1), "secondFunction");
        Yolan.addFunction(new ManyFunction(2), "thirdFunction");
        ....
    }
}
\end{lstlisting}

When implementing functions, it is also possible to create control structures, due to the laziness of Yolan objects. This is done by not necessarily calling the parameters \verb|value|-functions exactly one time each. The example below shows how the usual if-statement could be implemented:
\begin{lstlisting}
class YolanIf implements Function {
    Object apply(Yolan args[]) {
        // first evaluate the condition
        // and find out if it i true (not null)
        if(arg[0].value() != null) {
            // only evaluate the if the condition yields true
            return arg[1].value();
        } else {
            // only evaluate the if the condition yield false
            return arg[2].value();
        }
    }
}
\end{lstlisting}

\section{Summary}
