\chapter{Design}
\section{Design goals}

The goal is to make a scripting language that works both on modest mobile platforms as well as in web-based applications.
The language should be easy to embed and customise, and it should support powerfull abstractions like closures and higher order functions.
It should be possible to load new code at runtime, such that the language can also be used for scriptabel delivered content and configurations files.
To elaborate these are the more concrete design goals:
\begin{itemize}
\item Functions/closures as first-class values
\item Easy to embed and customise
\item Loading of code from source at runtime
\item Easy to learn, possibly for non-programmers
\item An execution model compatible with EcmaScript, such that it can easily be compiled to run in a browser environment
\item Usable on low end mobile devices. The minimal configuration would be a MIDP1/CLDC1.0 java enabled device with a jar-file limit is less than 64K. The jar-file contains the full application, where the scripting language is just a small part, so the size of the scripting language implementation should be even smaller
\end{itemize}
Especially the to last goals imposes a lot of constraints on the design.

%\section{Usage cases}

\section{Syntax}
The jar-file size limit imposes major constraints on the syntax: I did some experimental implementations that indicate that even a compact parser for an EcmaScript-like syntax would take in the magnitude of tens of kilobytes\footnote{Several prototypes were implemented, were the best result was through a space-optimised version of the Pratt-parser \cite{pratt, beautiful-code}. Parser generators usually use large tables, so these had to be handwritten}, which is too much, considering the jar-file limit. 
The options would then be to precompile the source into an intermediate code or to have a syntax which is very simple to parse, of which the latter is choosen, as the first goes against the design goal of being able to load source code at runtime. 

Some of the simplest syntaxes are those of Forth and Lisp-like language: Forth is essentially whitespace seperated words that are executed onto two stacks as they are read, whereas Lisp has almost explicitly written syntax tree that runs as a more common functional/imperative machine.
The Forth syntax matches a double stack machine and does not fit the EcmaScript exection model well, whereas the Lisp syntax is simpler to match with EcmaScript-like semantics. 
In addition the Forth syntax is often deemed difficult to work with, due to that the programmer needs to be innate aware of the execution stack, which is in stark opposition to Lisp where a lot of work has been done towards making it easy to teach to beginners, especially through the teachscheme project\cite{teachscheme}.
So a Lisp-like syntax is choosen to reduce the code footprint of the parser.

An alternative approach to a small footprint parser would be to follow the road of Smalltalk,
but that would still be more complex/larger than the Lisp-like approach.
Also the semantics has more focus on abstraction via a functions rather than objects, which makes Lisp-like syntax more natural than Smalltalk-like syntax. 
Focus on functions rather than objects for abstraction is partly to match some aspects of EcmaScript where for example function closures are used for information hiding as private properties does not exists (in ECMA-262)\cite{javascriptoop}. 

While the syntax is inspired from lisp,
there is a major difference, as the sequence type
in the EcmaScript machine model is very different from cons-cells.
So instead of lisp-lists, the syntax is represented
as a tree of arrays, so a lisp-like expression
\verb|(foo bar (baz quux))|
would match the EcmaScript data type of \verb|["foo", "bar", ["baz", "quux"]]|.
So the main structuring element is arrays rather then linked lists.
This also means that "." for cons does not make sense in the syntax.
To make the parser even smaller, it has less syntactic sugar than contemporary lisp-dialects, and most is delegated out into the macro system, as demonstrated later.

Lists are delimited by parenthesises and elements are white space seperated.
Whitespaces are space, tab, newline, and form-feed (``\verb| \t\n\r|'' as c-like escapes).
Numbers are written in decimal notation \verb|[0-9]*|.
Strings are enclosed in quotes ``\verb|"|'' and backslash ``\verb|\|'' is used for escape mechanism. Unicode characters in the range 0 to 255 can be escaped with an octal code as \verb|\[0-7][0-7][0-7]|. Quotes and backslash can be escaped as \verb|\"| and \verb|\\| respectively. 
Symbols and names can contain any characters except escape codes (ascii $<$ 32), space, !, ", \#, \$, \%, ', (, and ) which is either reserved or has other uses, and the may not start with a number. 
Comments starts with a semicolon ``\verb|;|'' and continues to the end of line or another escape symbol (ascii $<$ 32). Comments are only allowed after white spaces or parenthesises.
End of file closes all open parenthesis.

The choice of symbols in the different kinds of tokens is also influenced by what will simplify the tokenisation, and in some cases allow omission of checking assuming that the program is well formed.

The intermediate format of the parse tree also needs to be specified, as the macro system which is a part of the language works directly on it. 
The parse tree consists only of arrays and strings. String literals and numbers are represented as lists containing a "string" or "number" identifier and the literal itself, e.g.: ``\verb|(print "Hello " (+ 3 6))|'' is parsed into ``\verb|["print", ["string",| \verb|"Hello "],| \verb|["+", ["number", "3"], ["number", "6"]]]|'', and the macro system takes care of transforming the numbers into numeral values, giving more flexibility to the developer.

Detailed pseudocode for a full compact non-error-detecting\footnote{There is tradeoff between size and error-detection.} parser implementation is shown at figure~\ref{parsercode}. With minor changes this can be translated directly into C, Java and EcmaScript.
\begin{figure}
{\scriptsize
\begin{verbatim}
// Function that builds a syntax tree 
// from reading a sequence of characters
array parse() {
  // getchar returns next character or -1 on EOF
  int character = getchar(); 

  array result = [];
  while(true) {
    String str = "";

    // End of list/stream
    if(character == EndOfFile || character == ')') {
      return result;

    // Comment
    } else if(character == ';') {
        do {
          character = getchar();
        } while(character >= ' ');

    // List
    } else if(character == '(') {
        result.append(parse());
        character = getchar();

    // String
    } else if(character == '"') {
      character = getchar();
      while(character != '"' && character != EndOfFile) {
          if(character == '\\') {
              character = getchar();
              if('0' <= character && character <= '7') {
                  character = character * 64 + getchar() * 8 + getchar();
                  character -= '0' * 64 + '0' * 8 + '0';
              }
          }
          str.append(character);
          character = getchar();
      }
      character = getchar();
      result.append(["string", str]);

    // Number
    } else if('0' <= character && character <= '9') {
      do {
        str.append(character);
        character = getchar();
      } while('0' <= character);
      result.append(["number", str]);

    // Symbol or name
    } else if(character >= '*') {
      do {
        str.append(character);
        character = getchar();
      } while(character >= '*');
      result.append(str);

    // Whitespace
    } else {
      character = getchar();
    }
  }
}
\end{verbatim}
}
\caption{Detailed pseudocode for a compact parser without error detection}
\label{parsercode}
\end{figure}

\section{Compiler}

The execution model depends on the platform: on top of EcmaScript the code is translated into into an EcmaScript-program which is then evaluated, on low end java mobile devices it is executed through simple intepretation, and on a directly programmable device it can either be executed on a highly optimised interpreter or be JIT compiled.

An approach to make the implementation compact is only to implement few basic control structures and abstractions, and to that add a system for extending the language -- that is making the compiler an explicit and extendable part of the language.

The compiler contains a visitor function that traverse the syntax tree, and accumulate that which will become the executable code. It is by customising this visitor that the language can be extended. 
Instead having the developer rewrite the visitor function, then the visitor function just works as a dispatch where the developer can hook in new functionality.
The result is something akin to a simple non-hygeinic macro system. 

The system thus consists three parts: a visitor dispatch that finds the right visitor function based on the code, a set of visitor functions that takes a node in the syntax tree and appends the corresponding code to an accumulator/context-variable and finally an evaluator that executes the accumulated code.

As the syntax tree is written in the usual lisp for as a list where the first element is the operator name, the dispatch just need to lookup the operator name in a table of visitor functions and execute it, -- or alternatively call a default visitor function. The default visitor functions corresponds to an usual function-lookup/-call.
Pseudocode for the dispatch function can be seen in figure~\ref{visitor-dispatch}.


\begin{figure}
{\scriptsize
\begin{verbatim}
// collection of visitor functions
hashtable<string, function> macros;

CodeAccumulator visitorDispatch(CodeAccumulator acc, array node) {
    // handle empty lists
    if(length(node) == 0) {
        return nil;
    }

    // dispatch 
    macro = macros[node[0]];
    if(macro is not a function) {
        macro = macros["default-visitor-function"];
    }
    return macro(acc, node);
}
\end{verbatim}
}
\caption{Pseudocode for visitor dispatch}
\label{visitor-dispatch}
\end{figure}

Each visitor function is itself responsible for calling the dispatch function on the subtree, as this also allows quoting, optional compilation to be implemented as visitor functions
