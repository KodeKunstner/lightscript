\chapter{Design}
\section{Design goals}

The goal is to make a scripting language that works both on modest mobile platforms as well as in web-based applications.
The language should be easy to embed and customise, and it should support powerfull abstractions like closures and higher order functions.
It should be possible to load new code at runtime, such that the language can also be used for scriptabel delivered content and configurations files.
To elaborate these are the more concrete design goals:
\begin{itemize}
\item Functions/closures as first-class values
\item Easy to embed and customise
\item Loading of code from source at runtime
\item An execution model compatible with EcmaScript, such that it can easily be compiled to run in a browser environment
\item Usable on low end mobile devices. The minimal configuration would be a MIDP1/CLDC1.0 java enabled device with a jar-file limit is less than 64K. The jar-file contains the full application, where the scripting language is just a small part, so the size of the scripting language implementation should be even smaller
\end{itemize}
Especially the to last goals imposes a lot of constraints on the design.

\section{Usage cases}

\section{Syntax}
The jar-file size limit imposes major constraints on the syntax: I did some experimental implementations that indicate that even a compact parser for an EcmaScript-like syntax would take in the magnitude of tens of kilobytes\footnote{Several prototypes were implemented, were the best result was through a space-optimised version of the Pratt-parser \cite{pratt, beautiful-code}. Parser generators usually use large tables, so these had to be handwritten}, which is too much, considering the jar-file limit. 
The options would then be to precompile the source into an intermediate code or to have a syntax which is very simple to parse, of which the latter is choosen, as the first goes against the design goal of being able to load source code at runtime. 

Two classes of languages known for minimalistic syntax are Lisp-like languages and Forth-like language. 
