This chapter looks at the different mobile platforms for development.
To make it easier to get an overview, we divide them in three categories:
\begin{itemize}
\item{High end systems} cover smartphones, internet-tablets and PDAs.
These are have at least 32bit CPUs with FPU, 2M-128M of runtime memory.
It is here possible to write native applications for the phone, 
though they have to be digitally signed for distribution.

\item{Low end systems}
cover cheap phones, which still allows user downloadable applications. 
They typically have from 128K-4M runtime memory, 
and user programs can only be loaded as Java Midlets.
These are usually 32bit CPUs, and there may not be an FPU available. 

\item{Embedded systems}
cover those devices that are so small, that they do not support Java Midlets.
These devices do usually not allow user applications to be loaded.
Example of devices are very low end phones, washing machines and lego bricks.

\end{itemize}

The following sections will look at each of the three categories,
as a target for scripting language development. 

\section{High end systems}
\section{Low end systems}
\section{Embedded systems}


