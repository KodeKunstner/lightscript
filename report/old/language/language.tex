\chapter{Language}
\section{Language design}


The design criterias for the language is the following:
\begin{itemize}
\item Small language runtime, possibly piggybacking on host environment
\item Embeddable with easy to use foreign function interface
\item Portable - platform and host language independence
\item Small compiler
\end{itemize}




With every language feature

\begin{itemize}
\item First class functions
\item Object notation
\item Runtime loading of code
\item Garbage collection
\item EcmaScript like syntax
\end{itemize}

\subsection{Relation to EcmaScript}

\subsection{Object oriented programming}
prototyping
constructors
objects
this

\subsection{Functions}
closure and instantation

\subsection{Scope and the funarg problem}

Variables are either local to the current function, or global.
Inner functions cannot access locale variables of the enclosing function, which is a simplification compared to EcmaScript.

The reason is that if inner functions may contain references to locale variables, these variables could be accessed after the function has returned, and thus some of the local variables would have to be allocated on the heap rather than the stack.

Other options would be to allocate activation records on the heap or to do closure conversion.

A difficulty with closure conversion is that it is not sufficient to copy the local variable into the closure of the inner function, as the inner function might write back to the variable. Thus the local variable would have to be allocated on the heap, and both outer and inner function would need to access it through a reference, which would need slightly more complex compiler and virtual machine. 
Placing activation records on the heap is another solution, and can also have good performance, as investigated in \cite{appel94stack}. 
Unfortunately this depends on that there is a good garbage collector, which might not be the case on low end platforms. 
Heap allocation also has an extra memory overhead per activation record, in addition to the fragmentation which is also the case with the stack.

To eliminate these difficulties, the solution is only to have the current functions local scope and the global scope. 

Closures are possible by manually adding the variables to the function object.

\subsection{Type system}
NOTES:
I will start out with an untyped version of the language. 
When this is up and running, explicit typing and optimisations can be added.

\section{Language specification}

This section describes the MobyScript language.
The language is a subset of EcmaScript-3, augmented with some library function. 
This means that programs written in MobyScript also can be executed in EcmaScript-3 compliant interpreters after loading a function library.


\subsection{Types}
The language is dynamically typed, where each value has one of the eight following types.

\subsubsection{Undefined}

Undefined is the null type of MobyScript. It is the type of undefined values, either written as the literal \verb|undefined| and as uninitialised variables.

EcmaScript has seperate null type and undefined type, which is a contrast to MobyScript that only has the undefined type. 

\subsubsection{Boolean}
Boolean values has either the value \verb|true| or \verb|false|.

\subsubsection{Number}

Numbers in MobyScript are at least signed integers with 32 bit precision, and the behavior in case of overflow is unspecified. 
The motivation for this is that many mobile devices only have integer arithmetics.

This is a major difference from EcmaScript, where all numbers are 64 bit floating point numbers. 
MobyScript is still a subset of EcmaScript, as the arithmetic/logical operations of 64 bit floats yields the same result as on 32 bit integers, with the exception of overflow and division. This is why overflow handling is unspecified, and division/modulo-operators is a function rather than infix operators.

Numbers are entered in decimal notation.

\subsubsection{Strings}

\subsubsection{Array}
\subsubsection{Object}
\subsubsection{Function}
\subsubsection{Native}


\chapter{Language notes (TODO: cleanup)}
\section{Language elements}

\subsection{In the initial version}

\subsubsection{Variables and assignment}
\verb|var|
identifiers(\verb|A-Za-z_$|...)
\verb|this|
\verb|=|

\subsubsection{Subscripting}
\verb|[]|
\verb|.|

\subsubsection{Control flow}
\verb|if| \verb|else| \verb|while|

\verb|return|

\subsubsection{Comments}
\verb|/*|...\verb|*/|
\verb|//|
\subsubsection{Literals}
string-literals\verb|"...\" \n \t \\..."|
number-literals\verb|0123456789|
array-literals\verb|[]|
object-literals\verb|{}|
function-literals/lambda-expressions(\verb|function(|...\verb|)|)
\verb|true|
\verb|false|
\verb|undefined| (printed as null)

\subsubsection{Standard library}
\verb|delete|
\verb|parseInt|
array.\verb|push|
array.\verb|pop|
array.\verb|length|
array.\verb|join|
\verb|instanceof|
\verb|typeof|
\verb$||$ 
\verb|&&|
\verb|<|
\verb|>|
\verb|<=| 
\verb|>=|
\verb|===|
\verb|!==|
\verb|>|
\verb|+|
\verb|-|
\verb|!|

\subsubsection{Blocks}
blocks after while, if and functions \verb|{}|
statement separation \verb|;|

\subsubsection{Functions}
function-application\verb|()|
method-invocation\verb|()|


\subsection{Library extensions}
\verb|iterator|
\verb|has_type|
\verb|div|
\verb|mod|
\verb|assert|
\verb|log|
\subsubsection{Iterators}

Iterators are objects, used for iteration. An iterator contains a method \verb|next()| which returns true or false, depending on whether there are more elements, and goes to the next element. The \verb|next()| method also sets the \verb|val| property and possibly the \verb|key| property.

There is a function \verb|iterator(obj)| that creates an iterator across some kind of object. \verb|key| and \verb|val| is not set until \verb|next()| is called. To access all elements in an object or array:
\begin{verbatim}
iter = iterator(object);
while(iter.next()) {
    ... do something with iter.key and iter.val ...
}
\end{verbatim}
To access every character in a string:
\begin{verbatim}
iter = iterator(string);
while(iter.next()) {
    ... do something with iter.val ...
}
\end{verbatim}

The use of iterators is a pattern, which also applies to library routines such as input and output.



\subsection{Possible later expansions}

\subsubsection{Larger standard library}
\verb|<<|
\verb|>>|
\verb|>>>|
\verb$|$
\verb|^|
\verb|~|
\verb|&|
\verb|*|
Math.\verb|max|
Math.\verb|min|
string.\verb|fromCharCode|
string.\verb|charCodeAt(0)|
string.\verb|concat|
string.\verb|slice|
string.\verb|indexOf|
string.\verb|lastIndexOf|
string.\verb|length|
array.\verb|sort|
array.\verb|splice|
array.\verb|unshift|
array.\verb|concat|
array.\verb|reverse|
array.\verb|shift|
array.\verb|slice|
number.\verb|MAX_VALUE|
number.\verb|MIN_VALUE|
*\verb|.toString|
string.\verb|slice|
string.\verb|split|
string.\verb|toLowerCase|
string.\verb|toUpperCase|



\subsubsection{More control flow}
\verb|do|
\verb|for|
\verb|in|
\verb|? :|
\verb|switch|
\verb|case|
\verb|break|
\verb|default|
\verb|continue|


\subsubsection{Exceptions}
\verb|throw|
\verb|try|
\verb|catch|
\verb|finally|

\subsubsection{Arguments variable}
\verb|arguments|

\subsubsection{Eval}
\verb|eval|

\subsubsection{Syntactic sugar}
syntactic sugar not particularly interesting...
\verb|function|~name\verb|(|...\verb|)|
\verb|--|
\verb|++|
\verb|+=|
\verb|-=|
\verb|*=|
\verb$|=$
\verb|&=|
\verb|^=|
\verb|<<=|
\verb|>>=|
\verb|>>>=|

\subsection{Likely omitted:}
\subsubsection{Comparison with type coersion}
\verb|==|
\verb|!=|

\begin{verbatim}
"\n \t01e-0\t\r" == {"valueOf": function() { return true; } } 
\end{verbatim}
is true because in the comparison, the object is replaced by the result of the valueOf method, which is true. When true is casted into a number it is 1. When \verb|"\n \t01e-0\t\r"| is converted to a number it is also 1. Therefor the above object equals the above string. It should also be noted that this equals operator is not transitive, as the above object also equals \verb|"1"| which does not equals the above string.

\subsubsection{Floating point numbers}
number.\verb|toFixed|
number.\verb|toExponential|
number.\verb|toPrecision|
IEEE-754-numbers
\verb|parseFloat|
\verb|isNan|
\verb|isFinite|
\verb|/|
\verb|%|
\verb|/=|
\verb|%=|
\verb|NaN| \verb|Infinity|
\verb|Math|

\subsubsection{Random access to strings}
string.\verb|charAt|
string.\verb|charCodeAt(|n>0\verb|)|

\subsubsection{Automatic semi-colon insertion}
automatic-semicolon-insertion\verb|;|

\subsubsection{Regular Expressions}
regular-expressions\verb|/ /|
string.\verb|match|
string.\verb|replace|
string.\verb|search|

\subsubsection{Inheritance and object constructors}
\verb|prototype|
\verb|new|
Constructor: \verb|Date|, \verb|Object|, \verb|Function|, \verb|Array|, \verb|String|, \verb|Boolean|, \verb|Number|, \verb|RegExp|, \verb|Error|, \verb|EvalError|,\verb|RangeError|,\verb|ReferenceError|,\verb|SyntaxError|,\verb|TypeError|, \verb|URIError|,

\subsubsection{Nonessential library functions}
string.\verb|substring|
string.\verb|toLocaleLowerCase|
number.\verb|toLocaleString|
string.\verb|toLocaleUpperCase|
string.\verb|localeCompare|
unary\verb|+|
Date.*
\verb|decodeURI|
\verb|decodeURIComponent|
\verb|encodeURI|
\verb|encodeURIComponent|

\subsubsection{Other language features}
\verb|void|
\verb|null|
\verb|with|
comma-expression\verb|,|
blocks-anywhere\verb|{}|

\section{Datatypes}

\subsection{Undefined}
... undefined and null...
\subsection{Boolean}
... true and false ... what is true/false ... 
\subsection{Number}
... integer ... div ... decimal notation only ...
\subsection{Strings}
... immutable ... sequential ... hashable...
\subsection{Array}
... not objects ...
\subsection{Object / Hashtable}
... oop, with constructor-functions...
\subsection{Function}
... scope... first class

\section{Abstract Intermediate Syntax Tree}
\subsection{The machine}
\subsection{Types}
\subsubsection{Undefined}
\subsubsection{Boolean}
\subsubsection{Number}
\subsubsection{Strings}
\subsubsection{Array}
\subsubsection{Object}
\subsubsection{Function}
\subsubsection{Native}
\subsection{The nodes}

\subsubsection{Comment}

\subsubsection{While}
\subsubsection{If-else}
\subsubsection{Statement block}
\subsubsection{GetVar}
flagged if upval/downval/local/global
\subsubsection{SetVar}
flagged if upval/downval/local/global
\subsubsection{Literal}
strings, numbers
\subsubsection{Functionclosure}
contains variable information: scope = {var: (upval | downval | local) ... }
impl: byte[] code, Object[] consts, Ref[] refs, Function(code, consts, refs), 
python-like-$this$-as-an-argument
\subsubsection{Return}
\subsubsection{Sequential logical operator}
\verb|&&| \verb#||#
\subsubsection{ApplyFunction/method}

\subsection{Functions}
\subsubsection{Get}
a[b], a.b
\subsubsection{Put}
a[b]= c; a.b = c
\subsubsection{New array}
\verb|[]|
\subsubsection{New Object}
\verb|{}|
\subsubsection{True}
\subsubsection{False}
\subsubsection{Undefined}
\subsubsection{Delete}
\subsubsection{Array.push}
\subsubsection{Array.length}
\subsubsection{Array.pop}
\subsubsection{Array.join}
\subsubsection{isa}
joined typeof/instanceof
\subsubsection{!}
\subsubsection{===}
\subsubsection{$<$}
\subsubsection{+}
\subsubsection{-}
\subsubsection{parseInt}
