\chapter{* Introduction}
\section{Project description NB:SYNOPSIS-NEEDS-SYNC}
    The topic of the project is to design and implement a scripting language
that runs on low-end mobile devices. This is both to create a practical tool, and
also a focus for exploration of programming language theory. 

\begin{comment}
The motivation is that a scripting language makes it is easier to make applications for mobile 
devices, and that existing freely available scripting languages
are very limited, slow, or simply does not run on the low-end mobile devices.
\end{comment}

    The educational goals are to learn about programming language design and
implementation, and to learn about programming on mobile devices. Through the
project, I should be able to evaluate and choose programming language features
and implementation techniques, and design and implement a scripting language.

    The focus of the language is that it should be portable, embeddable and have
a low memory footprint. Portable implies that it should run on a varity of devices,
from very low-end mobile phones to high-end computers, possibly also within a web browser. 
Embeddable implies that it should be easy to include within, and interface with
other applications. Low memory footprint implies that it should be suitable for
running on platforms where the available memory is measured in kilobytes rather
than megabytes. 

    The approach will be pragmatic and favor simplicity.

\section{Motivation}

The motivation is that a scripting language could make it is easier to make applications for mobile devices.
Scripting languages for low end mobile devices is a niche, 
where a new language may actually be of practical use:
existing freely available scripting languages either does not run on low-end mobile devices, or they are very slow or limited.

The restrictions of low end mobile devices also imposes some challenges, that may lead to interesting solutions.


From a personal point of view, 
I would like to get started on development for mobile devices, 
and would also like to brush up on programming language implementation.
Design and implementation of a scripting language for mobile devices is spot on this topic.


\section{*The structure and content of the report}

The report starts out with some background information: this introduction, and a survey on topics related to the project in chapter~\ref{survey}.
Then the methodology and design for the project is elaborated in chapter~\ref{approach}, including some of the principal design choices for the language implementations.
The next couple of chapters are then the actual results: the LightScript language in chapter~\ref{lightscript}, the Yolan language in chapter~\ref{yolan}, and some benchmarking of the two new languages compared to existing scripting languages in chapter~\ref{benchmark}.
Finally ther is a discussion of the results in chapter~\ref{discussion}, rounding off with a conclusion in chapter~\ref{conclusion}
