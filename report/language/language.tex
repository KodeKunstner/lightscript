\chapter{The Language}

\section{...}
\subsection{Numbers}
Numbers in MobyScript are integers with at least 32 bit precision.
The result of overflow are not defined.
Platforms might not support floating point arithmetic, examples are Java CLDC 1.0 devices and devices without FPU programmed in C.
This is a major deviation from EcmaScript where all numbers are 64 bit floating point. 
Fortunately the operations on 64 bit floats yields the same result as operations on 32 bit integers, except for division and overflow. 
Thus the division operator \verb|\| is not included in the MobyScript language, and
instead a new library function for integer division \verb|div| is introduced.

To simplify parsing, numbers can only be entered in decimal notation.

\subsection{Strings}

A string is a sequence of characters, where a character can be any unicode symbol. 

In EcmaScript strings supports random access, and contains unicode characters in range 0...65545.
This makes good sense if implemented as an array of 16-bit numbers. Unfortunately that representation use TODO:fix-spelling:excessive memory and does not support all unicode characters. 
By limiting the string access to be sequential, it is possible to support all unicode characters, while having a more compact format. 
This also makes i possible to add compression later to the strings in memory.
Strings in MobyScript can therefore only be accessed sequentially, for example through iteration.

To simplify parsing, strings can only be quoted with \verb|"|, and the only escape symbols are \verb|\\| and \verb|\"|.
