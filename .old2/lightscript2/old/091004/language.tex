\documentclass[10pt]{article} 
%\usepackage[paperheight=220mm,paperwidth=170mm]{geometry}
%\usepackage[paperheight=150mm,paperwidth=200mm]{geometry}
%\usepackage{geometry}
\usepackage{a4} 
\usepackage{makeidx}
%\usepackage[danish]{babel} 
\usepackage[utf8]{inputenc}
\usepackage{textcomp}
\usepackage{amsmath}
\usepackage{amssymb}
\usepackage{amsthm}
\usepackage{graphicx} 
\usepackage{verbatim} 
\usepackage{fancyhdr}
\usepackage{listings} 
%\usepackage[colorlinks,pagebackref]{hyperref}
\usepackage[colorlinks, linkcolor=black, anchorcolor=black, citecolor=black, men
ucolor=black, pagecolor=black, urlcolor=black]{hyperref}
%\usepackage{backref}
\usepackage{url}
\frenchspacing
\makeindex
\pagestyle{plain}
\lstset{language=python}
\lstset{escapeinside=~~}
%\addtolength{\voffset}{-40pt}
%\addtolength{\textheight}{100pt}

% node in tt
\newcommand{\n}[1]{{\tt #1}}

% children in tt em
\newcommand{\ch}[1]{{\tt\em #1}}

% properties em
\newcommand{\p}[1]{{\em #1}}

% curly brackets
\newcommand{\lb}[0]{\verb|{|}
\newcommand{\rb}[0]{\verb|}|}

\title{LightScript$^{2}$}
\author{Rasmus Jensen\footnote{rasmus@lightscript.net}}
\begin{document}

\maketitle
\section{Language introduction}
\subsection{Design motivation}
\subsection{Scope}
\subsection{Syntaxes}
\subsubsection{LightScript}
\subsubsection{MoPy}
\subsection{Backends}
\subsubsection{JavaScript 1.7}
\subsubsection{EcmaScript}
\subsubsection{Virtual machine}
7-register strongly typed virtual machine for easy compilation to native x84 and arm-thumb. register 0 is used for state-pointer on arm, and stack-pointer on x86.
register 1/2/3 are for parameter passing/return, and thus implicit source and destination for some operations


types: any, string, int, float, object, array, function
\paragraph{Java implementation}
\paragraph{C implementation}
\paragraph{Native code gen}
Remember: type tags can be tested with shifting + conditional...

\subsubsection{Java - compiled}
\section{Abstract syntax tree nodes}
\subsection{Object properties}
\subsubsection{\em nodetype}
The kind of AST node, for example \n{for} loop, \n{this} reference, or \n{string} literal. 
\subsubsection{\em datatype}
Static type information for the node. Not yet used, but will be in future versions.
\subsubsection{\em children}
Child nodes 
\subsubsection{\em comment}
Comment applicable to this node and following sibling nodes until next comment. This will typically be rendered as:
\begin{verbatim}
    ... precedingSiblingNode;

    // text in
    // comment
    statementFromThisNode;
    statementFollowingSiblingNode;
\end{verbatim}
with an empty line before the comment.

\subsubsection{Position: \em begin end}
\p{begin} and \p{end} refer to the position of the beginning and end of this node in a parsed file. Used for error reporting, an may be dummy values, if not a parsed file.
The positions are objects, TODO: define position objects.

\subsection{Superobject properties}
\subsubsection{\em version}
The language version of the node, currently ``2.0.0''.

\subsection{Control flow}
\subsubsection{\tt begin \em stmt1 stmt2 ...}
Sequenced statements, semantically equivalent to \verb|{| \ch{stmt1} \verb|;| \ch{stmt2} \verb|;| ... \verb|}|
\subsubsection{\tt for \em element sequence statements}

For each \ch{element} in the \ch{sequence} object, execute the \ch{statements}.
The \ch{sequence} is enumerated similarly to Python, such that the \ch{element}s are the individual values of a list, or the keys of a dictionary. 

In new versions of JavaScript, the semantic equivalent would be: {\tt for(\ch{element} in LightScriptIterator(\ch{sequence})) \ch{statements}},
which in old versions of JavaScript can be implemented as: 
{\tt try \verb|{| {\em iter} = LightScriptIterator({\em sequence}); for(;;) \verb|{| 
    {\em element} = {\em iter}.next();
    {\em statements}
\verb|}| \verb|}| catch(e) \verb|{| if(e !== StopIteration) throw e; \verb|}|}, where \ch{iter} is a local temporary variable name.

\subsubsection{\tt and \em stmt1 stmt2}
Equivalent to: \ch{stmt1} \verb|&&| \ch{stmt2}
\subsubsection{\tt or \em stmt1 stmt2}
Equivalent to: \ch{stmt1} \verb]||] \ch{stmt2}
\subsubsection{\tt while \em cond body}
Equivalent to: \verb|while(|\ch{cond}\verb|)| \ch{stmt2}
\subsubsection{\tt cond \em cond1 body1 ... condn bodyn elsebody}
Conditional, if-elif-else. Last \ch{elsebody} is optional, and if omitted there wille be no else branch.
This is equivalent to:
\begin{verbatim}
if(cond1) {
    body1
} else if( ...
...
} else if(condn) {
    bodyn
} else {
    elsebody
}
\end{verbatim}
\subsubsection{\tt try \em body varname except-body}
Equivalent to: \verb|try{| \ch{body} \verb|}catch(| \ch{varname} \verb|){| \ch{except-body} \verb|}|
Notice that varname may only be used locally within \ch{except-body} and not anywhere else in the local scope.
\subsubsection{\tt return \em value}
Equivalent to: \verb|return| \ch{value}
\subsubsection{\tt throw \em value}
Equivalent to: \verb|throw| \ch{value}
\subsubsection{\tt function \em params locals globals body}
\subsubsection{\tt set \em name value}

\subsection{Literal values}
\subsubsection{\tt array \em val1 val2 ...}
\subsubsection{\tt dictionary \em key1 val1 key2 val2 ...}
\subsubsection{Simple values: \tt string number} 
{\tt string}, {\tt number}
\subsubsection{Constant values: \tt true false undefined}
\n{true}, \n{false}, \n{undefined}

\subsection{References}
\subsubsection{\tt varname}
\p{scope}
\subsubsection{Special values: \tt this global}
\n{this} \n{global}
\subsubsection{\tt nil}
\subsubsection{\tt get} 
Array/object/string subscripting
\subsubsection{\tt put} 
Subscript assignment

\subsection{Builtin functions}
The following operations first evaluate all their arguments, and then execute the function and return the result. They can all be used as expressions.
\subsubsection{Unary functions}
Unary functions are usually written with prefix notation, for example \verb|!False|.
\paragraph{\tt clone} Prototypal object creation
\paragraph{\tt not} Logical negation
\paragraph{\tt neg} Arithmetic negation
\paragraph{\tt len} Get the length of an array/string/... use this instead of \verb|blah.length|, to avoid magic properties.
\paragraph{\tt str} Convert to string
\paragraph{\tt num} Convert to number
\subsubsection{Binary functions}
Binary functions are usually written with infix notation, for example \verb|6 * 7|.
\paragraph{\tt add} Numeral addition or string join. Notice that LightScript requires the operands to have same type for easier future updates to the languag
\paragraph{\tt sub mul div rem} Subtraction, multiplication, floating division, or remainder, only for numbers
\paragraph{\tt less leq geq greater} Comparison, operands must have same type and be either strings or numbers
\paragraph{\tt eq neq} Equality test, works for any type

\section{JavaScript code definitions used}
\verbatiminput{ls2lib.js}

\end{document}
