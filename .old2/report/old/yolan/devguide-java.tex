This document is an introduction and reference to the Java implementation of Yolan.
Yolan is an extension language, which means that it is designed to be embedded into other applications.
The focus of this implementation is to reduce the size added to the JAR file, for which motivation is that the JAR size is the most limiting factor on low-end mobile devices, and that the language should be able to be embedded on large applications on those devices.

Features of the Java implementation:
\begin{itemize}
\item Adds less than 5KB to the JAR size, including builtin library
\item Only single-threaded/non-reentrant to achive the small size
\item Works with Java Mobile Edition/J2ME, requires only CLDC 1.0/MIDP 1.0
\item Builtin types maps directly to standard java classes, to make integration easier
\item Parse and execute one expression at a time, not needing entire program in memory and allowing interactive programming 
\end{itemize}

To reduce the JAR footprint, the entire implementation is done in one class \verb|Yolan| and one interface \verb|Function|. 
The \verb|Yolan| class can both be instantiated into delayed computations, which is elaborated below, and also contain a single static runtime environment. While a single language runtime limits some applications, it also reduces memory usage, each code node does not need a pointer to which runtime it is connected. 
Yolan objects are pieces of code that can be evaluated, e.g. delayed computations.
The \verb|Function| interface is what an object needs to implement to be callable from Yolan. Essentially it needs a function that takes a number of delayed computations as parameters, and then return the value. In this sense Java functions callable from Yolan are actually lazy, and themselves responsible for evaluating their arguments.

\section{Getting started}

The core method of the Yolan objects is the \verb|value()| method, which evaluate the code the Yolan object represents, and return the result.
Notice that this method may both throw exceptions and errors if the code the object represents has faults, so if we are executing user coder, or want to be robust against errors in script, the Yolan evaluation can be surrounded by a \verb|catch(Throwable)|.

Yolan objects are created with the \verb|readExpression| method that parses the next Yolan expression from an input stream. So if we want to create a simple interactive interpreter, we can write the code:
{\scriptsize 
\begin{verbatim}
class Main {
    public static void main(String [] args) throws java.io.IOException {
        Yolan yl = Yolan.readExpression(System.in);
        while(yl != null) {
            try {
                System.out.println("Result: " + yl.value().toString());
            } catch(Throwable yolanError) {
                System.out.println("Error: " + yolanError.toString());
            }
            yl = Yolan.readExpression(System.in);
        }
    }
}
\end{verbatim} 
}
in a file called \verb|Main.java|, place it in the same directory as Yolan.class and Function.class, compile it, and then we have an interpreter, where Yolan expressions can be evaluated interactively.

Notice that the input stream \verb|System.in| can be replaced with any input stream, so it is the same basic idea for evaluating files, programs as strings within the application, or even as streams across the network, where Yolan could work as a shell for remote scripting/controlling an application.

If we want to execute an entire stream, there is a short hand builtin method for doing that: \verb|eval|, so for example:
{\scriptsize 
\begin{verbatim}
class Main {
    public static void main(String [] args) throws java.io.IOException {
        Yolan.eval(new FileInputStream(new File("script.yl")));
    }
}
\end{verbatim}
}
would open the file "script.yl", and evaluate all the expressions within it. 
\verb|eval| throws away the results of the individual expressions and does not print them,
so the above code is only if we have added some user defined functions to Yolan that allows it to do something practical.

\subsection{Adding functions to the runtime}
It is necessary to add some functions to the scripting language in order to make it do something practical. While the core library supports basic data structures etc., it does not have any kind of input/output as that is platform dependent.
So when the language needs communicate with the user, or work on the state of the actual application, there needs to be added some functionality, which is easiest via adding some functions to the runtime.

This is where the \verb|Function| interface comes into play, - it contains a single function \verb|apply|, that takes an array of Yolan objects as parameter, and returns an object.
Notice that the parameters are passed lazily, e.g. they are only evaluated when the called function chooses to evaluate them, so we need to call the \verb|value| when we want the actual value.
In order to add a new java function to be callable from the runtime, the method \verb|Yolan.addFunction| takes a name as a string and a \verb|Function| as parameters, and binds the name to the function. An example would be to make add an new function println to the runtime, which takes one argument, which it prints out to standard out:
{\scriptsize 
\begin{verbatim}
class PrintingFunction implements Function {
    Object apply(Yolan args[]) {
        System.out.println(args[0].value());
    }
}
class Main {
    public static void main(String [] args) throws java.io.IOException {
        Yolan.addFunction("println", new PrintingFunction);
        Yolan.eval(new FileInputStream(new File("script.yl")));
    }
}
\end{verbatim}
}
Now the above program, reads and evaluate the file script.yl, with an augmented runtime which also has the println function.

\subsection{Values and types}
The builtin types in yolan is mapped to Java classes for easier interoperbility:TODO:spelling:,
so lists matches java.util.Stack, dictionaries matches java.util.Hashtable, strings matches java.lang.String, nil/false matches the value null, integers matches java.lang.Integer, and iterator matches java.util.Enumeration. 
Operations on those data types are just as the native builtin types. 

Adding other data types is done by just adding functions that works on those data types, as Yolan just runs with the builtin java object system.

\subsubsection{Functions defined in Yolan}
When a user defines functions in Yolan, they are instances of the Yolan class. 
Before calling them, the number of arguments can be found using the \verb|nargs| method.
If the Yolan object isn't a callable user defined function, the result of \verb|nargs()| is -1, which also can be used to check if a Yolan object is a callable function.
The function is then applied with the \verb|apply| method, which takes the parameters to the function as parameters, for example:
{\scriptsize 
\begin{verbatim}
...
    // evaluation some yolan object that yields a function
    Yolan function = yl.value();
    // ensure that it is a function and it takes two arguments
    if(function.nargs() == 2) {
        // apply the function 
        result = function.apply(arg1, arg2);
    } else ...
....
\end{verbatim}
}

The apply method is defined from zero, up to three parameters. If there is the need for apply methods with more arguments, they are easy to add, see section~\ref{j2mesource} for more details.

\subsection{Modifying the runtime}

In order for the scripting language to be practical, it should be able to work and share data with the application. 
Of course this can be done with functions, and evaluation, as described above, but an additional connection with the language can be added by accessing the variables of the runtime.
For this there are three functions: \verb|Yolan.resolveVar|, \verb|Yolan.getVar|, and \verb|Yolan.setVar|.

When a value from the runtime is accessed, there need to be a handle, which is found with \verb|resolveVar|. This handle can then be used for reading and writing the variable. The motivation for the handle is that it takes time to lookup what a variable name, so this computation can be done once for each variable that needs to be accessed, and then additional accesses to the resolved variabled is significant faster. The handle itself is just an integer, and \verb|resolveVar| just takes the variable name as a string as parameter, and return the handle. If the variable does not exist in the runtime, space is allocated for it.

With a handle, it is then possible to set the value of a variable with \verb|setVar|, for example setting the variable foo to 42 can be done with:
{\scriptsize 
\begin{verbatim}
    Yolan.setVar(Yolan.resolveVar("foo"), new Integer(42));
\end{verbatim}
}
and similarly the variable can be read with \verb|getVar|:
{\scriptsize 
\begin{verbatim}
    Object result = Yolan.getVar(Yolan.resolveVar("foo"));
\end{verbatim}
}

If it the variable is commonly accessed, it saves time to cache the handle across calls, a la:
{\scriptsize 
\begin{verbatim}
class Class {
    int fooHandle;
    Class() {
        fooHandle = Yolan.resolveVar("foo");
    }

    int someMethod() {
        ... perhaps some scripts modifying foo is executed ...
        Object foo = Yolan.getVar(fooHandle);
        ...
    }

    void otherMethod() {
        ... 
        Yolan.setVar(fooHandle, "A literal value or some variable");
        ...
    }
}
\end{verbatim}
}

When defining functions, as described earlier, it is actually the same that is happening, where the function is encapsulated in a Yolan object and added to the runtime as with \verb|setVar|.
\subsection{Calling back functions}
apply nargs
\subsection{Resetting the runtime and saving space}
When the scripting language is only used in some part of the application, it can be pratical to be able to unload the runtime data in order to save memory. This can be done with \verb|Yolan.wipe()|, after which, the references in the runtime are set to zero, allowing data to be garbage collected.
When the runtime is wiped, existing Yolan expressions can no longer be evaluated, and trying to evaluate them will yield errors. Also notice that it is necessary to reset the runtime before any of it can be used.

Resetting the runtime can be done with \verb|Yolan.reset()|. When the runtime is reset, all variable handles are invalidated, all variables are removed from the runtime, and only the builtin functions are added. Existing Yolan expressions are also invalidate, and evaluation of those may lead to unexpected behavior. User defined functions and variables needs to be re-added.
Resetting is practical when scripts are run, after each other, and must mess the runtime up for each other.

\section{More on adding functions to the runtime}

The simple approach to adding functions to the runtime would require a new class for each function, which will add significantly to the size of the Jar-file.
If the code size is critical, then this can often be reduced by combining the functions in a single class, for example via a switch dispatch, e.g.:
{\scriptsize 
\begin{verbatim}
class ManyFunction implements Function {
    int id;
    ManyFunction(int id) {
        this.id = id;
    }
    Object apply(Yolan args[]) {
        switch(id) {
            case 0: // first function
                    ....
                break;
            case 1: // second function
                    ....
                break;
            case 2: // third function
                    ....
                break;
            ....
            default:
                throw SomeKindOfError();
    }
    static void register() {
        Yolan.addFunction(new ManyFunction(0), "firstFunction");
        Yolan.addFunction(new ManyFunction(1), "secondFunction");
        Yolan.addFunction(new ManyFunction(2), "thirdFunction");
        ....
    }
}
\end{verbatim}
}

When implementing functions, it is also possible to create control structures, due to the lazyness of yolan objects. This is easily done by not calling all of the parameters \verb|value| functions once. The example below, shows how the usual if-statement could be implemented:
{\scriptsize 
\begin{verbatim}
class YolanIf implements Function {
    Object apply(Yolan args[]) {
        // first evaluate the condition
        // and find out if it i true (not null)
        if(arg[0].value() != null) {
            // only evaluate the if the condition yields true
            return arg[1].value();
        } else {
            // only evaluate the if the condition yield false
            return arg[2].value();
        }
    }
}
\end{verbatim}
}

\section{Hacking the source}
\label{j2mesource}
While the implementation only consist of one java-class-file, 
this is actually several distinct classes, joined to save space.

The core of the implementation is the the delayed computation objects that can also matches the parser tree. Each object just contains an index to some executable code\footnote{dispatched via a switch, as java does not support pointers to code}, and some data - its a kind of closure. 

The interpreter is just walking and executing the parsed tree. Well, mostly - during the execution it does som simple optimisations and caching: an example is that if a node for a variable name is encountered, the execution of that node replaces the node itself with a node that references the cell that contains the variable, and then evaluate that new node.

The runtime object is implemented as a set of static functions and variables.
This limits the implementation to a single excution environment, but but also reduces the memory usage, as there are no need for passing pointers to the runtime object around.

The parser is implemented a static function, that builds a parse tree from an input stream.

When changing the builtin objects, there need to be changes three different places: There are the defines for the id matching a certain piece of code, then there is the interpreter dispatch, and finally there is the initialisation of the object into the runtime.


Most of the use of the scripting language can be done without going into the details of the implementation. Even support floating point numbers and so on, can be added via the \verb|Function| interface.
