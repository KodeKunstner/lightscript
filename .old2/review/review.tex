\documentclass{article} 
\title{Review - NXTalk article}
\begin{document}
\maketitle

\section*{Overall impression}
The paper is not ground breaking, but I do think it is
worthwhile, sound, and also quite interesting
to read, so I would definitely recommend it.

There already exists other compact Smalltalk implementations,
but this adds something new by opening more up towards
end-user programming via the integration with Squeak.

\section*{Summary}
The paper describes NXTalk which is a Smalltalk implementation targeting the Lego-NXT embedded platform.

It starts with background the project, including a nice introduction to the Lego NXT brick.

Next it describes the NXTalk virtual machine in great details.
Focus is on low memory usage rather than execution speed.
Memory usage is reduced by using 16 bit per ``word'' which is either an 15 bit signed integer or a 15 bit id into an object table. 
The object table then contains packed pointers to objects and meta data such as reference count.
Memory allocation is best fit with a linear time search when allocating/deallocating large objects.
Garbage collection is done with a combination of reference counting and mark-and-compact.
Method lookup is first tried on reciever object, then in a hash'ed cache and if not found there, then af full superclass lookup is done.
Execution is done via a bytecode stack machine. 
All opcodes are 1 byte long, and includes jump lengths and literal indices which limits the number of literals, instance variables etc. 
Threading is a part of the virtual machine and is using preemptive green threads.

Then there is a short description of the NXTalk programming environment, which is integrated with squak, but using seperate libraries, and has a packager for transfering the applications onto the NXT.
It is also possible to simulate NXTalk within squeak.

The next is an evaluation of the vm, where the system uses 33KB flash and 7-9KB ram. Some experiments with compiler flags are also documented. A full GC cycle takes about 60ms, and it seems like the performance is approximately 1000 bytecodes per 270 ms.

A survey of similar projects then follows. - it focusses both on other languages on the NXT and on other compact Smalltalk implementations. 

Finally there are some suggestions towards future works. 
Especially it sounds like this is the first steps towards more a beginner friendly robotic programming, through future integration with etoys.

\section*{Questions and suggestions}
Some questions and suggestions:

\begin{itemize}
\item
With regard to the bytecode interpreter, it
could have been elaborated a bit how it is implemented,
- is it token threaded, or a switch-statement or some
more complex decoding exploiting the packed representation?
\item
Java seems to be mentioned as a dynamic language, (page 1 and 10)
- dynamic in which sense?
\item
Regarding the evaluation, it would have been interesting
to see how it compares to other systems, for example:
microbenchmarks vs. other languages on the NXT (pbLua etc.),
space usage of bytecodes, vs. intermediate code representation
in other smalltalk systems.
Another measure could be how many cycles per executed bytecode.
Instead I think that there might be a bit too much focus on the effects of compiler flags (e.g. effects of thumb-instructions and optimisations), which might be omitted.
\item
Perhaps also mention why a stack based virtual machine is used rather than a register based one.
\item
I think section 3.2.* could be elaborated whereas section 3.1.* might be a bit to detailed.
\item
Figure 6 is not obvious to read
\item
It would be nice with sections summarizing the exact differences between NXTalk and smalltalk, e.g.: other standard library(?), limited numbers of instance variables and literals, etc.
\item
Under future works, it seems like this is the first step towards integrating robotic programming with existing a beginner-friendly smalltalk programming environment. If this is the case, it would probably be nice to have more emphasis on that earlier in the report so the direction/motivation is more clear.
\end{itemize}

\end{document}

